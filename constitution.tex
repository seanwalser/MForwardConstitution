\documentclass{constitution}
\usepackage{mathpazo}
\usepackage{lineno}
\usepackage{xspace}
\usepackage{soul}
\usepackage{fancyhdr}
\usepackage{color}
\usepackage{graphicx}

%\newcommand{\temp}[1]{{\color{red} #1}}
\newcommand{\temp}[1]{\ul{\bf #1}\todo{TBD}\xspace}

\addtolength{\parskip}{\baselineskip}

\rhead{\thepage}
\chead{\qquad\qquad Article \Roman{article}}
\lhead{UMich Student Constitution (Rev: 1141)}
\rfoot{}
\cfoot{}
\lfoot{}

\begin{document}
	\renewcommand{\thepage}{\roman{page}}
	\title{Constitution of the Student Body of the Ann Arbor Campus of the University of Michigan}
	\author{\includegraphics[width=5in]{S4PGlogo}}
	\date{Revision 1141 (9 Feb 2010)}
	\maketitle
	\tableofcontents
	\newpage

	\renewcommand{\thepage}{\arabic{page}}
	\setcounter{page}{1}
	\pagestyle{fancy}
	\headheight 35pt
%	\thispagestyle{empty}
	\linenumbers
	
	\begin{center}
		\Large Constitution of the Student Body of the Ann Arbor Campus\\ of the University of Michigan
	\end{center}
	
	\begin{center}
		\bfseries Preamble
	\end{center}

	AN EDUCATED CITIZENRY being indispensable to the preservation of our civic rights and liberties; creating, securing, and applying knowledge and wisdom among the people being the chief mission of our university; and active participation in our own education being imperative to the success of these undertakings;

	    we, the students of the University of Michigan's Ann Arbor campus, hereby establish this Constitution

	to promote academic freedom and responsibility, foster fellowship and collaboration among the Students, and guarantee a public forum for Student expression.
	

	\input{governance}

	\input{legislative}
	
	\input{executive}
	
	\input{judicial}
	
	%\input{elections}
	
	\input{amendment}
	
	\input{organizations}

	\input{initiatives}
	
	\input{studentrights}

\end{document}
